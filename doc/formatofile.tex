\documentclass{scrartcl}

\usepackage[utf8]{inputenc}
\usepackage[italian]{babel}
\begin{document}
\section{Note}

\paragraph{Vincoli 9-10} I vincoli 9 e 10 richiedono l'utilizzo di un $M$. Per questo
valore è stato scelto $10^6$, ma probabilmente con altre istanze il numero deve essere cambiato.


\paragraph{Formato di file}
Il formato di input è il seguente:
\begin{itemize}
	\item la prima riga contiene le dimensioni della scatola $(S^1, S^2, S^3) = (W, D, H)$
	\item la seconda contiene $N$ il numero di ogetti
	\item Seguono $N$ righe con il seguente formato:
	\begin{itemize}
		\item $s^0$ nella rotazione numero $0$
		\item $s^1$ nella rotazione numero $0$
		\item $s^2$ nella rotazione numero $0$
		\item Massa
		\item Profitto
	\end{itemize}
\end{itemize}

\paragraph{Note sui constraint}
Di seguito vengono riportate alcune modifiche ai constraint in modo da portare
tutte le variabili nella parte sinistra della dis/equazione.
\begin{itemize}
	\item 
	Il constraint numero (7) è stato riscritto come:
	$$
	\begin{array}{l}
	\sum_{\delta \in \Delta}(b_{ij}^\delta + b_{ji}^\delta) \geq t_i + t_j - 1 \iff \\
	\sum_{\delta \in \Delta}(b_{ij}^\delta + b_{ji}^\delta) - t_i - t_j \geq - 1 \iff \\
	+ b_{ij}^1 + b_{ji}^1 + b_{ij}^2 + b_{ji}^2 + b_{ij}^3 + b_{ji}^3 - t_i - t_j \geq  -1
	\end{array}
	$$
	\item 
	Il vincolo numero (9) è stato riscritto come:
	$$
	\begin{array}{l}
	\chi_i^\delta + \sum_{r \in R} s_{ir}^\delta \rho_{ir} \leq \chi_j^\delta + M(1 - b_{ij}^\delta)\\
	\chi_i^\delta + (\sum_{r \in R} s_{ir}^\delta \rho_{ir}) -  \chi_j^\delta + M b_{ij}^\delta \leq M
	\end{array}
	$$
	\item 
	Il vincolo numero (10) è stato riscritto come:
	$$
	\begin{array}{l}
	\chi_j^\delta + \sum_{r \in R} s_{jr}^\delta \rho_{ir} \leq \chi_j^\delta + M(1 - b_{ji}^\delta)\\
	\chi_j^\delta + (\sum_{r \in R} s_{jr}^\delta \rho_{ir}) -  \chi_i^\delta + M b_{ji}^\delta \leq M
	\end{array}
	$$
\end{itemize}

\subsection{Per aggiungere i balancing constraint (14/15)}
Bisogna implementare il metodo \verb|calculateGamma| che dovrebbe calcolare i valori di $\gamma_{ir}^\delta$ nella classe \verb|Instance3BKP| e  
e inoltre aggiungere nel parsing del file due righe extra che contengono i valori
$L^0, L^1, L^2$ e $U^0, U^1, U^2$.

Per usare la versione estesa del programma si deve usare l'opzione \verb|--extended| o per brevità \verb|-e|.
\end{document}