\documentclass{scrartcl}

\usepackage[utf8]{inputenc}
\usepackage[italian]{babel}
\begin{document}
\section{Variabili}
\begin{itemize}
	\item $z_k$ assume valore $1$ se lo zaino $k-$simo viene utilizzato, $0$ altrimenti.
	\item $F_k$ è un costo fisso per utilizzare lo zaino $k-$simo.
	\item $L$ è un fattore per cui viene moltiplicato il primo componente della funzione obiettivo per contare
	di più rispetto al minimizzare l'uso degli zaini.
\end{itemize}
	
\begin{equation}
\left( \sum_{k \in K} \sum_{j \in J} L \cdot \frac{p_{j}}{p_{min}} \cdot t_{kj} \right) - \left( \sum_{k \in K} F_k \cdot z_k \right)
\end{equation}
s.t. 
\begin{equation}
\sum_{j \in J} w_j d_j h_j t_{kj} \leq W_kD_kH_k \quad\quad \forall k \in K 
\end{equation}
\begin{equation}
\sum_{\delta\in\Delta}(b_{kij}^\delta + b_{kji}^\delta) \geq t_{ki} + t_{kj} - 1, \quad i < j,\ k \in K,\ i \in J,\ j \in J
\end{equation}
\begin{equation}
\chi_{ki}^\delta + \sum_{r \in R} s_{ir}^\delta \rho_{ir} \leq S_k^\delta \quad \quad k \in K,\ i \in J,\ \delta \in \Delta
\end{equation}
\begin{equation}
\chi_{ki}^\delta + \sum_{r \in R} s_{ir}^\delta \rho_{ir} \leq \chi_{kj}^\delta + M(1 - b_{kij}^\delta), \quad \quad i < j,\ k \in K, i \in J, j \in J, \delta \in \Delta
\end{equation}
\begin{equation}
\chi_{kj}^\delta + \sum_{r \in R} s_{jr}^\delta \rho_{jr} \leq \chi_{ki}^\delta + M(1 - b_{kji}^\delta), \quad \quad i < j,\ k \in K, i \in J, j \in J, \delta \in \Delta
\end{equation}
\begin{equation}
\chi_{ki}^\delta \leq M t_{ki}\quad\quad k \in K, i \in J, \delta \in \Delta
\end{equation}
\begin{equation}
b_{kij}^\delta \leq t_{ki}\quad \quad k \in K, i \in J, j \in J, \delta \in \Delta
\end{equation}
\begin{equation}
b_{kji}^\delta \leq t_{kj} \quad \quad k \in K, i \in J, j \in J, \delta \in \Delta
\end{equation}
\begin{equation}
\sum_{r \in R} \rho_{ir} = \sum_{k \in K} t_{ki}\quad \quad i \in J
\end{equation}
\begin{equation}
\label{constraint:multi:tkzk}
t_{kj} \leq z_k\quad \quad k \in K, j \in J
\end{equation}
\begin{equation}
\label{constraint:multi:onlyInOneKnapsack}
\sum_{k \in K} t_{kj} \leq 1 \quad \quad j \in J
\end{equation}
\begin{equation}
\chi_{ki}^\delta \geq 0 \quad \quad k \in K, i \in J, \delta \in \Delta
\end{equation}
\begin{equation}
t_{ki} \in \{0,1\} \quad \quad k \in K, i \in J
\end{equation}
\begin{equation}
b_{kij}^\delta \in \{0,1\} \quad \quad k \in K, i \in J, j \in J, \delta \in \delta
\end{equation}
\begin{equation}
\rho_{ir} \in \{0,1\} \quad\quad i \in J, r \in R
\end{equation}
\begin{equation}
z_k \in \{0,1\} \quad \quad k \in K
\end{equation}

\section{Note}
\begin{itemize}
\item Tutti i vincoli a parte \ref{constraint:multi:tkzk} e \ref{constraint:multi:onlyInOneKnapsack} sono stati ottenuti adattando i vincoli del modello di partenza.
\item Il vincolo \ref{constraint:multi:tkzk} fa sì che un oggetto possa essere inserito nel $k-$simo zaino soltanto se lo zaino viene effettivamente utilizzato nella soluzione.
\item Il vincolo \ref{constraint:multi:onlyInOneKnapsack} fa sì che un oggetto possa essere inserito al massimo in uno zaino. 
\end{itemize}
\end{document}
