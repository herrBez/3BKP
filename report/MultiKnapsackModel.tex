\documentclass{scrartcl}
\usepackage{amsmath}
\usepackage{amssymb}
\usepackage{xcolor}
\usepackage[utf8]{inputenc}
\usepackage[italian]{babel}


\newcommand{\MutliLineEquation}[3]{ 
	\ensuremath{
		#1 = \left\{
		\begin{array}{ll}
		1& \text{#2} \\
		0& \text{#3} \\
		\end{array}
		\right.
	}
}

\title{Estensione con più Zaini}
\begin{document}
\maketitle
Lo scopo di questo documento è fornire un estensione del modello con più zaini introdotto in Baldi et al. \cite{Baldi20129802}.

Dato un insieme di zaini $K = \{1, \dots, m\}$, le loro dimensioni $S_k^0, S_k^1, S_k^2$ e il loro costo associato $f_k$ (modella quanto costa usare lo zaino $k$) e un insieme di oggetti $J = \{1, \dots, n\}$ di profitto $p_j$ e dimensione $w_j, d_j, h_j$.
Si cercano $K' \subseteq K$ e $J' \subseteq J$ tali che il profitto sia massimo.


\textbf{Disclaimer: Non sono stati considerati i Balancing constraints del paper originale}.

\section{Modello}
\begin{itemize}
	\item $J$ è l'insieme degli oggetti da inserire nello zaino di cardinalità $n$, cui sono associati gli indici $i$ e $j$;
	\item $K$ è l'insieme di possibili zaini, cui è associato l'indice $k$;
	\item $\Delta$ l'insieme delle dimensioni $\{1,2,3\}$ con l'indice associato $\delta$;
	\item $R$ l'insieme delle rotazioni degli oggetti di cardinalità $6$ e indice $r$;
	\item $s_{ir}^\delta$ dimensione dell'oggetto $i$--simo lungo la dimensione $\delta$ quando l'oggetto è rotato nella rotazione $r$;
	\item $S_{k}^\delta$ la dimensione del $k$--simo zaino nella dimensione $k$;
	\item $\chi_{ki}^\delta$ la coordinata del punto più in fondo, in basso a sinistra dell'oggetto $i$ lungo la dimensione $\delta$ nello zaino $k$;
	\item \MutliLineEquation{t_{ki}}{Se l'oggetto $i$ è contenuto nello zaino $k$}{Altrimenti}
	\item \MutliLineEquation{b_{kij}^\delta}{Se l'oggetto $i$ viene prima dell'oggetto $j$ lungo la dimensione $\delta$ nello zaino $k$}{Altrimenti}
	\item \MutliLineEquation{\rho_{ir}}{Se l'oggetto $i$ è rotato con la rotazione $r$}{Altrimenti}
	\item \MutliLineEquation{z_k}{Se lo zaino $k$--simo viene utilizzato}{Altrimenti}
	
\end{itemize}



\section{Funzione Obiettivo - Con spiegazioni}
Vengono descritti i parametri che compaiono nella funzione obiettivo
\begin{itemize}
	\item $f_k$ è il costo associato all'utilizzo del $k-$simo zaino e viene letto dal file di input.
	\item $L$ è un fattore per cui viene moltiplicato il primo componente della funzione obiettivo per contare
	di più rispetto alla minimizzazione dei valori delle coordinate.
	Siccome deve essere più grande di qualsiasi valore $\chi_{ki}$ si può prendere per questo valore
	$$
	L = max(S_{k}^\delta : k \in K : \delta \in \Delta) + 1
	$$
	se si rendono tutti i valori $f_k$ e $p_i$ maggiori o uguali a $1$.

	\item Di default $\Delta' = \Delta$, ma per rendere il modello più veloce si può cercare di ottimizzare rispetto una sola delle dimensioni, per esempio $y$ ($\Delta' = \{2\}$),	o nessuna ($\Delta = \varnothing$). In quest'ultimo caso
	la seconda componente della funzione obiettivo viene ignorata.
\end{itemize}


\begin{equation}
L \cdot \left(\left( \sum_{k \in K} \sum_{j \in J} \bar{p_{j}} \cdot t_{kj} \right) - \left( \sum_{k \in K} \bar{f_k} \cdot z_k \right) \right)
- \left( \sum_{k \in K}\sum_{i \in J}\sum_{\delta \in \Delta'} \chi_{ki}^\delta \right)
\end{equation}

dove:
\begin{itemize}
	\item $$\bar{p_{j}} = \frac{p_{j}}{g}$$
	\item $$\bar{f_{k}} = \frac{f_{k}}{g}$$
	\item dove 
	$$
	g = min(p_j\ :\ j \in J, f_k\ :\ k \in K)
	$$
\end{itemize}
In modo tale che tutti i valori $\bar{f_k}$ e $\bar{p_j}$ siano maggiori o uguali ad $1$ e quindi se moltiplicati
con $L$ pesino più dell'ottimizzazione delle coordinate.

Naturalmente questo metodo assume che i profitti degli oggetti e i costi associati agli zaini siano positivi e \emph{non nulli}.
Questa condizione può essere forzata a run-time con un \textit{assert}.


\section{Vincoli}
\begin{equation}
\sum_{j \in J} w_j d_j h_j t_{kj} \leq W_kD_kH_k \quad\quad \forall k \in K 
\end{equation}

\begin{equation}
\sum_{\delta\in\Delta}(b_{kij}^\delta + b_{kji}^\delta) \geq t_{ki} + t_{kj} - 1, \quad i < j,\ k \in K,\ i \in J,\ j \in J
\end{equation}
\begin{equation}
\chi_{ki}^\delta + \sum_{r \in R} s_{ir}^\delta \rho_{ir} \leq S_k^\delta \quad \quad k \in K,\ i \in J,\ \delta \in \Delta
\end{equation}
\begin{equation}
\chi_{ki}^\delta + \sum_{r \in R} s_{ir}^\delta \rho_{ir} \leq \chi_{kj}^\delta + M(1 - b_{kij}^\delta), \quad \quad i < j,\ k \in K, i \in J, j \in J, \delta \in \Delta
\end{equation}
\begin{equation}
\chi_{kj}^\delta + \sum_{r \in R} s_{jr}^\delta \rho_{jr} \leq \chi_{ki}^\delta + M(1 - b_{kji}^\delta), \quad \quad i < j,\ k \in K, i \in J, j \in J, \delta \in \Delta
\end{equation}
\begin{equation}
\chi_{ki}^\delta \leq M t_{ki}\quad\quad k \in K, i \in J, \delta \in \Delta
\end{equation}
\begin{equation}
b_{kij}^\delta \leq t_{ki}\quad \quad k \in K, i \in J, j \in J, \delta \in \Delta
\end{equation}
\begin{equation}
b_{kji}^\delta \leq t_{kj} \quad \quad k \in K, i \in J, j \in J, \delta \in \Delta
\end{equation}
\begin{equation}
\sum_{r \in R} \rho_{ir} = \sum_{k \in K} t_{ki}\quad \quad i \in J
\end{equation}
\begin{equation}
\label{constraint:multi:tkzk}
t_{kj} \leq z_k\quad \quad k \in K, j \in J
\end{equation}
\begin{equation}
\label{constraint:multi:onlyInOneKnapsack}
\sum_{k \in K} t_{kj} \leq 1 \quad \quad j \in J
\end{equation}
\begin{equation}
\chi_{ki}^\delta \geq 0 \quad \quad k \in K, i \in J, \delta \in \Delta
\end{equation}
\begin{equation}
t_{ki} \in \{0,1\} \quad \quad k \in K, i \in J
\end{equation}
\begin{equation}
b_{kij}^\delta \in \{0,1\} \quad \quad k \in K, i \in J, j \in J, \delta \in \delta
\end{equation}
\begin{equation}
\rho_{ir} \in \{0,1\} \quad\quad i \in J, r \in R
\end{equation}
\begin{equation}
\label{zk:in:0:1}
z_k \in \{0,1\} \quad \quad k \in K
\end{equation}

\subsection{Note}
\begin{itemize}
\item Tutti i vincoli a parte \ref{constraint:multi:tkzk}, \ref{constraint:multi:onlyInOneKnapsack} e \ref{zk:in:0:1} sono stati ottenuti adattando i vincoli del modello di partenza.
\item Il vincolo \ref{constraint:multi:tkzk} fa sì che un oggetto possa essere inserito nel $k-$simo zaino soltanto se lo zaino viene effettivamente utilizzato nella soluzione.
\item Il vincolo \ref{constraint:multi:onlyInOneKnapsack} fa sì che un oggetto possa essere inserito al più in uno zaino. 
\item Il vincolo 9 è ridondante rispetto ad 8. (Ogni vincolo creato con 9 ha il corrispettivo in 8).
\item Per il vincolo 7 si può utilizzare invece di un generico $M$, l'$L$ definito per la funzione obiettivo.
\end{itemize}

\section{Dimensioni del problema}
Assumendo verosimilmente che 
\begin{itemize}
	\item $|K| < |J|$
\end{itemize}  
dalla tabella \ref{table:no:variables}
si ottiene che il numero di variabili è minore di $3|J|^3 + 4|J|^2 + 7|J|$, quindi
nell'ordine di $O(|J|^3)$.


\begin{table}[h!]
	\center
	\begin{tabular}{|l|l|}
		\hline
		Variabili & \# Variabili \\
		\hline
		& \\
		$z_k$ & $|K|$ \\
		$\rho_{ir}$ & $|J| \cdot 6$\\
		$t_{ki}$ & $|J| \cdot |K|$ \\
		$X_{ki}^\delta$ & $|J| \cdot |K| \cdot | \Delta |$\\
		$b_{kij}^\delta$ & $|J| \cdot |J| \cdot |K| \cdot |\Delta|$\\ 
		& \\
		\hline
		& $3|J||J||K| + 4|J||K|+6|J|+K$ \\
		\hline
	\end{tabular}
	\caption{Numero di variabili del problema.}
	\label{table:no:variables}
\end{table}


Nella tabella \ref{table:no:constraints} sono riportati Il numero di vincoli per ogni tipo di vincolo. 
Assumendo come per le variabili che $|K| < |J|$ il numero di vincoli può essere sovrapprossimato in circa:
$$
\frac{19}{2}|J|^3 + \frac{15}{2}|J|^2 + 10|J|
$$
che è sempre nell'ordine $O(|J|^3)$.


\begin{table}[h!]
	\center
\begin{tabular}{|l|l|}
	\hline
	Vincoli & \# Vincoli\\
	\hline
	(2) & $|K|$ \\
	(3) & $|K| |J| (|J|-1)/2$ \\
	(4) & $3\cdot |K| |J| $ \\
	(5) & $3\cdot |K| |J| (|J|-1)/2$\\
	(6) & $3\cdot |K| |J| (|J|-1)/2$ \\
	(7) & $3\cdot |K| |J|$ \\
	(8) & $3\cdot |K| |J| |J|$\\
	(9) - È ridondante & 0 \\
	(10)& $|J|$ \\
	(11)& $|K| |J|$ \\
	(12)& $|J|$ \\
	\hline
	Totale    & $3\cdot|K||J||J| + 7\cdot|K||J|(|J|-1)/2 + 7|K||J|+ 2|J| + |K|$ \\
	(senza definizione dei domini) & \\
	\hline
	Dominio & \\
	\hline 
	(13)& $3 \cdot |K| |J|$ \\
	(14)& $|K| |J|$ \\
	(15)& $3 \cdot |K| |J| |J|$ \\
	(16)& $6 |J|$ \\
	(17)& $|K|$ \\
	\hline
	Totale & $6\cdot|K||J||J| + 7\cdot|K||J|(|J|-1)/2 + 11|K||J|+ 8|J| + 2|K|$ \\
	\hline
\end{tabular}
	\caption{Numero di vincoli per ogni tipo di vincolo.}
	\label{table:no:constraints}
\end{table}


\section{File di Input}
\begin{itemize}
	\item $K$, il numero di zaini
	\item Seguono $K$ righe contenente nella riga $k$ i dati per lo zaino $k$ intervallati da uno spazio:
	\begin{itemize}
		\item $S_k^1$
		\item $S_k^2$
		\item $S_k^3$
		\item $f_k$
	\end{itemize} 
	\item $N$, il numero di oggetti ($|J|$)
	\begin{itemize}
		\item $s_i^1$
		\item $s_i^2$
		\item $s_i^3$
		\item $m_i$
		\item $p_i$
	\end{itemize}
\end{itemize}



\bibliographystyle{unsrt}

\bibliography{sources}

\end{document}
