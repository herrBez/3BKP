\section{Estensione del 3KP con più zaini}
\label{sec:extension1}

Siano dati:
\begin{itemize} 
	
	\item un insieme di parallelepipedi $K = \{1, \dots, m \}$\ 
	(gli \emph{zaini}) le loro dimensioni $W_k$, $H_k$, $D_k$, riferite
	anche come $S_k^0$, $S_k^1$, $S_k^2$ e il loro costo 
	associato $f_k$ (modella quanto costa usare lo zaino $k$);
	\item un insieme di parallelepipedi $J = \{1, \dots, n\}$\ 
	(gli \emph{oggetti}) di profitto $p_j$ e dimensione $w_j$, $h_j$, $d_j$.
\end{itemize}

Si cercano $K' \subseteq K$ e $J' \subseteq J$ tali che gli oggetti in $J'$ 
siano contenuti nei limiti degli zaini $K'$ e il profitto sia massimo.


\textbf{N.B.} Qui viene analizzato un modello senza i vincoli di 
bilanciamento.
Il loro adattamento è stato considerato nella sezione 
\ref{sec:Balancing:Constraint}.


\subsection{Modello}
\label{sec:modello:multi3BKP}
\begin{itemize}
	\item $J$ è l'insieme degli oggetti -- di cardinalità $n$ -- cui sono 
	associati gli indici $i$ e $j$;
	\item $K$ è l'insieme di possibili zaini, cui è associato l'indice $k$;
	\item $\Delta$ l'insieme delle dimensioni $\{0,1,2\}$ con l'indice associato
	$\delta$;
	\item $R$ è l'insieme delle possibili rotazioni degli oggetti, ha 
	cardinalità $6$ e indice $r$;
	\item $s_{ir}^\delta$ dimensione dell'oggetto $i$--simo lungo la dimensione
	$\delta$ quando l'oggetto è ruotato con la rotazione $r$;
	\item $S_{k}^\delta$ la dimensione del $k$--simo zaino nella dimensione 
	$\delta$;
	\item $\chi_{ki}^\delta$ la coordinata del punto più in fondo, in basso a 
	sinistra  (il punto di coordinate minime) dell'oggetto $i$ lungo la 
	dimensione $\delta$ nello zaino $k$;
	\item \MutliLineEquation{t_{ki}}{Se l'oggetto $i$ è contenuto nello 
	zaino $k$}{Altrimenti}
	\item \MutliLineEquation{b_{kij}^\delta}{Se l'oggetto $i$ viene prima 
	dell'oggetto $j$ lungo la dimensione $\delta$ nello zaino $k$}{Altrimenti}
	\item \MutliLineEquation{\rho_{ir}}{Se l'oggetto $i$ è ruotato con la 
	rotazione $r$}{Altrimenti}
	\item \MutliLineEquation{z_k}{Se lo zaino $k$--simo viene 
	utilizzato}{Altrimenti}
	
\end{itemize}



\subsubsection{Funzione Obiettivo}
\begin{equation}
\text{ max }\left(\sum_{k \in K} \sum_{i \in J} p_i \cdot t_{ki}\right) - 
\left(\sum_{k \in K} f_k \cdot z_k\right)
\end{equation}


\subsubsection{Vincoli}
\begin{flalign}
\label{eq:constraint:volume}
\sum_{j \in J} w_j d_j h_j t_{kj} \leq W_kD_kH_k && \forall k \in K \\
%
\sum_{\delta\in\Delta}(b_{kij}^\delta + b_{kji}^\delta) \geq t_{ki} + t_{kj} - 1 && i < j,\ k \in K,\ i \in J,\ j \in J \\
%
\label{eq:constraint:coodinateLessThanSkdelta}
\chi_{ki}^\delta + \sum_{r \in R} s_{ir}^\delta \rho_{ir} \leq S_k^\delta \cdot t_{ki} + M (1 - t_{ki}) && k \in K,\ i \in J,\ \delta \in \Delta \\
%
\chi_{ki}^\delta + \sum_{r \in R} s_{ir}^\delta \rho_{ir} \leq \chi_{kj}^\delta + M(1 - b_{kij}^\delta) && i < j,\ k \in K, i \in J, j \in J, \delta \in \Delta \\
%
\chi_{kj}^\delta + \sum_{r \in R} s_{jr}^\delta \rho_{jr} \leq \chi_{ki}^\delta + M(1 - b_{kji}^\delta) && i < j,\ k \in K, i \in J, j \in J, \delta \in \Delta \\
%
\chi_{ki}^\delta \leq M t_{ki} && k \in K, i \in J, \delta \in \Delta \\
%
\label{constraint:bkij:leq:tki}
b_{kij}^\delta \leq t_{ki} && k \in K, i \in J, j \in J, \delta \in \Delta \\
%
\label{constraint:bkji:leq:tki}
b_{kji}^\delta \leq t_{kj} && k \in K, i \in J, j \in J, \delta \in \Delta \\
%
%b_{kij}^\delta \leq z_k && k \in K, i \in J, j \in J, \delta \in \Delta \\
%
\label{eq:constraint:rho:ir:leq:sumk}
\sum_{r \in R} \rho_{ir} = \sum_{k \in K} t_{ki} && i \in J \\
%
\label{constraint:multi:tkzk}
t_{kj} \leq z_k && k \in K, j \in J \\
%
\label{constraint:multi:onlyInOneKnapsack}
\sum_{k \in K} t_{kj} \leq 1 && j \in J \\
%
\chi_{ki}^\delta \geq 0 && k \in K, i \in J, \delta \in \Delta \\ 
%
t_{ki} \in \{0,1\} && k \in K, i \in J \\
%
b_{kij}^\delta \in \{0,1\} && k \in K, i \in J, j \in J, \delta \in \Delta \\
%
\rho_{ir} \in \{0,1\} && i \in J, r \in R \\
%
\label{zk:in:0:1}
z_k \in \{0,1\} && k \in K \\
\nonumber
\end{flalign}

\subsubsection{Note}
\begin{itemize}
\item Tutti i vincoli a parte (\ref{constraint:multi:tkzk}), 
(\ref{constraint:multi:onlyInOneKnapsack}) e (\ref{zk:in:0:1}) sono stati 
ottenuti adattando i vincoli del modello di partenza.
\item Il vincolo (\ref{constraint:multi:tkzk}) fa sì che un oggetto possa essere
 inserito nel $k$--simo zaino soltanto se lo zaino viene effettivamente 
 utilizzato nella soluzione.
\item Il vincolo (\ref{constraint:multi:onlyInOneKnapsack}) fa sì che un oggetto
possa essere inserito al più in uno zaino. 
\item Il vincolo (\ref{constraint:bkji:leq:tki}) è ridondante rispetto a 
(\ref{constraint:bkij:leq:tki}), infatti ogni vincolo di tipo
(\ref{constraint:bkji:leq:tki}) ha un corrispettivo di tipo 
(\ref{constraint:bkij:leq:tki}).
\item Al posto di un generico $M$ si potrebbe usare un 
parametro $E$ così definito:
$$
E = max(S_k^\delta : k \in K, \delta \in \Delta) + 1
$$ 
\item Nelle versioni precedenti di questo documento era stata fornita una
versione sbagliata del vincolo (\ref{eq:constraint:coodinateLessThanSkdelta}).

Questa faceva sì che, ogni oggetto inserito in uno zaino, 
dovesse essere contenuto nei limiti di \emph{tutti} gli zaini, perfino
quelli che non venivano utilizzati nella soluzione.
Questa nuova versione fa in modo che gli oggetti siano sottoposti a vincoli
del tipo (\ref{eq:constraint:coodinateLessThanSkdelta}) \emph{solo}
se l'oggetto è effetivamente contenuto nello zaino $k$, ovvero se 
$t_{ki}$ è uguale a 1.

\end{itemize}


\subsection{Dimensioni del problema}
\label{sec:orig:dimensioneDelProblema}
Assumendo verosimilmente che 
\begin{itemize}
	\item $|K| < |J|$
\end{itemize}  
dalla Tabella \ref{table:no:variables}
si ottiene che il numero di variabili è minore di $3|J|^3 + 4|J|^2 + 7|J|$, 
quindi nell'ordine di $O(|J|^3)$.



\begin{table}[h!]
	\center
	\begin{tabular}{|l|l|}
		\hline
		Variabili & \# Variabili \\
		\hline
		& \\
		$z_k$ & $|K|$ \\
		$\rho_{ir}$ & $|J| \cdot 6$\\
		$t_{ki}$ & $|J| \cdot |K|$ \\
		$\chi_{ki}^\delta$ & $|J| \cdot |K| \cdot | \Delta |$\\
		$b_{kij}^\delta$ & $|J| \cdot |J| \cdot |K| \cdot |\Delta|$\\ 
		& \\
		\hline
		& $3|J||J||K| + 4|J||K|+6|J|+ |K|$ \\
		\hline
	\end{tabular}
	\caption{Numero di variabili del problema.}
	\label{table:no:variables}
\end{table}


Nella Tabella \ref{table:no:constraints} sono riportati il numero di vincoli 
per ogni tipo di vincolo. 
Assumendo, come per le variabili che $|K|$ sia minore di $|J|$, il numero di 
vincoli può essere sovrapprossimato nel modo seguente:
$$
\frac{19}{2}|J|^3 + \frac{15}{2}|J|^2 + 10|J|
$$
che è sempre nell'ordine $O(|J|^3)$.


\begin{table}[h!]
	\center
\begin{tabular}{|l|l|}
	\hline
	Vincoli & \# Vincoli\\
	\hline
	(2) & $|K|$ \\
	(3) & $|K| |J| (|J|-1)/2$ \\
	(4) & $3\cdot |K| |J| $ \\
	(5) & $3\cdot |K| |J| (|J|-1)/2$\\
	(6) & $3\cdot |K| |J| (|J|-1)/2$ \\
	(7) & $3\cdot |K| |J|$ \\
	(8) & $3\cdot |K| |J| |J|$\\
	(9) - È ridondante & 0 \\
	(10)& $|J|$ \\
	(11)& $|K| |J|$ \\
	(12)& $|J|$ \\
	\hline
	Totale    & $3\cdot|K||J||J| + 7\cdot|K||J|(|J|-1)/2 + 7|K||J|+ 2|J| + |K|$ \\
	(senza definizione dei domini) & \\
	\hline
	Dominio & \\
	\hline 
	(13)& $3 \cdot |K| |J|$ \\
	(14)& $|K| |J|$ \\
	(15)& $3 \cdot |K| |J| |J|$ \\
	(16)& $6 |J|$ \\
	(17)& $|K|$ \\
	\hline
	Totale & $6\cdot|K||J||J| + 7\cdot|K||J|(|J|-1)/2 + 11|K||J|+ 8|J| + 2|K|$ \\
	\hline
\end{tabular}
	\caption{Numero di vincoli per ogni tipo di vincolo.}
	\label{table:no:constraints}
\end{table}

\subsection{Problema Accessorio}
\label{sec:multi:problemaAcessorio}
Una volta risolto il problema, potrebbe essere desiderabile cercare di ridurre 
i valori delle coordinate
$\chi$ il più possibile.

Il modello rimane lo stesso ma cambia la funzione obiettivo
\begin{equation}
\text{ min } \sum_{k\in k}\sum_{i \in J}\sum_{\delta \in \Delta'} \chi_{ki}^\delta
\end{equation}
dove $\Delta' \subseteq \Delta$ e $\Delta' \neq \varnothing$.

Inoltre si fissano tutte le variabili $z_k, \rho_{ir}, t_{ki}$ e tutte le 
variabili $b_{kij}^\delta$ con $z_k = 0$, ma non quelle con $z_k = 1$, perché 
$b_{kij}^\delta$ definisce delle precedenze molto forti tra oggetti -- molte 
volte non necessarie in alcune direzioni -- che potrebbero limitare 
l'ottimizzazione delle variabili $\chi$: per questo è stato preferito non 
fissarle.

Una breve discussione sulle performance di qualità della soluzione e tempo
necessario per risolvere il problema fissando o meno le variabili $b$ è fornita
alla Sezione \ref{sec:benchmark:accessorio}.

