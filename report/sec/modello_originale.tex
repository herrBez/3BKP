\section{Modello Originale}
\label{sec:modello:originale}

Viene qui riportato il modello originale descritto in Baldi et al. 
\cite{Baldi20129802}
con gli aggiustamenti motivati nella Sezione~\ref{sec:refusi}.
Per ora vengono ignorati i modelli di bilanciamento che verranno discussi
nella Sezione \ref{sec:Balancing:Constraint}.




Sia dato:
\begin{itemize} 
	\item uno zaino di dimensione $W, H, D$, anche riferite come $S^0, S^1, S^2$.
	\item un insieme di parallelepipedi $J = \{1, \dots, n\}$\ (gli \emph{oggetti})
	di profitto $p_j$ e dimensione $w_j, h_j, d_j$.
\end{itemize}

Si cerca e $J' \subseteq J$ tale che gli oggetti in $J'$ siano contenuti nei 
limiti dello zaino.


\subsection{Modello}
\begin{itemize}
	\item $J$ è l'insieme degli oggetti -- di cardinalità $n$ -- cui sono 
	associati gli indici $i$ e $j$;
	\item $\Delta$ è l'insieme delle dimensioni $\{0,1,2\}$ con l'indice associato
	$\delta$;
	\item $R$ è l'insieme delle possibili rotazioni degli oggetti, ha 
	cardinalità $6$ e indice $r$;
	\item $s_{ir}^\delta$ dimensione dell'oggetto $i$--simo lungo la dimensione
	$\delta$ quando l'oggetto è ruotato con la rotazione $r$;
	\item $S^\delta$ la dimensione dello zaino nella dimensione $\delta$;
	\item $\chi_{i}^\delta$ la coordinata del punto più in fondo, in basso a 
	sinistra (il punto di coordinate minime) dell'oggetto $i$ lungo la 
	dimensione $\delta$ nello zaino;
	\item 
	\MutliLineEquation{t_{i}}{Se l'oggetto $i$ è contenuto nello zaino}{Altrimenti}
	\item 
	\MutliLineEquation{b_{ij}^\delta}{Se l'oggetto $i$ viene prima dell'oggetto 
	$j$ lungo la dimensione $\delta$ nello zaino}{Altrimenti}
	\item \MutliLineEquation{\rho_{ir}}{Se l'oggetto $i$ è ruotato con la 
	rotazione $r$}{Altrimenti}
\end{itemize}


\subsubsection{Funzione Obiettivo}
\begin{flalign}
\text{ max } \left(\sum_{i \in J} p_i \cdot t_{i}\right)
\end{flalign}

\subsubsection{Vincoli}
\begin{flalign}
\label{eq:constraint:volume}
\sum_{j \in J} w_j d_j h_j t_{j} \leq \prod_{\delta \in \Delta} S^\delta & \\
%
\sum_{\delta\in\Delta}(b_{ij}^\delta + b_{ji}^\delta) \geq t_{i} + t_{j} - 1 &&  i < j,\ i \in J,\ j \in J \\
%
\label{eq:orig:chi+s_ir:leq:SdeltaK}
\chi_{i}^\delta + \sum_{r \in R} s_{ir}^\delta \rho_{ir} \leq S^\delta && i \in J,\ \delta \in \Delta \\
%
\chi_{i}^\delta + \sum_{r \in R} s_{ir}^\delta \rho_{ir} \leq \chi_{j}^\delta + M(1 - b_{ij}^\delta) &&  i < j,\ i \in J, j \in J, \delta \in \Delta \\
%
\label{eq:orig:error:10}
\chi_{j}^\delta + \sum_{r \in R} s_{jr}^\delta \rho_{jr} \leq \chi_{i}^\delta + M(1 - b_{ji}^\delta) && i < j,\ i \in J, j \in J, \delta \in \Delta \\
%
\chi_{i}^\delta \leq M t_{i} &&  i \in J, \delta \in \Delta \\
%
\label{eq:orig:constraint:bij:leq:ti}
b_{ij}^\delta \leq t_{i} && i \in J, j \in J, \delta \in \Delta \\
%
\label{eq:orig:constraint:bji:leq:tj}
b_{ji}^\delta \leq t_{j} && i \in J, j \in J, \delta \in \Delta \\
%
\setcounter{equation}{15}
%
\label{eq:orig:constraint:rho:ir:leq:sumk}
\sum_{r \in R} \rho_{ir} = 1 && i \in J \\
%
\chi_{i}^\delta \geq 0 && i \in J, \delta \in \Delta \\
%
t_{i} \in \{0,1\} && i \in J \\
%
b_{ij}^\delta \in \{0,1\} && i \in J, j \in J, \delta \in \Delta \\
%
\rho_{ir} \in \{0,1\} && i \in J, r \in R \\
\nonumber
\end{flalign}

\subsection{Nota}
\begin{itemize}
\item Per rendere la lettura di questo documento più facile e i riferimenti al 
modello
del paper originario più semplici (soprattutto nella sezione \ref{sec:refusi}),
si è preferito usare la stessa numerazione di Baldi et al.~\cite{Baldi20129802}.
\item I vincoli \ref{eq:orig:balancing1} e \ref{eq:orig:balancing2}
 ovvero i vincoli di bilanciamento, vengono qui 
ignorati deliberatamente. Verranno brevemente discussi
nella sezione~\ref{sec:Balancing:Constraint}.

\end{itemize}
