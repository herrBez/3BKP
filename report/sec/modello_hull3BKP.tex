\section{Problema alternativo}
\label{sec:extension2}
Un problema alternativo che è  stato analizzato è quello di considerare le 
dimensioni degli zaini come delle variabili, con un upper bound fissato 
($S_k^0, S_k^1, S_k^2$), in cui lo scopo è quello di trovare le dimensioni 
minime che gli zaini devono avere, affinché possano contenere tutti gli oggetti.

\subsection{Modello}
Il modello differisce di poco dal modello fornito nella
Sezione \ref{sec:modello:Multi3BKP}, quindi vengono elencate soltanto le 
differenze:
\begin{itemize}
	\item $S_k^\delta$ definisce l'upper bound della lunghezza dello zaino $k$ 
	lungo la dimensione $\delta$.
	
	\item $\sigma_k^\delta$ è una variabile che codifica l'estensione dello 
	zaino $k$ lungo la dimensione $\delta$. È limitato superiormente dal valore
	$S_k^\delta$ fornito in input.
	Qualora si volesse trovare le dimensioni di un unico zaino che racchiuda 
	tutti gli oggetti $J$ è sufficiente	mettere in input uno zaino solo e 
	fissare gli upper bound a valori molto alti. 
	
	\item Al posto della variabile $\rho_{ir}$ qui viene definita una variabile
	che itera anche tra gli zaini: $\rho_{kir}$. 
	Se si mantiene la $\rho$ su due dimensioni e si utilizza il vincolo 
	(\ref{eq:constraint:rho:ir:leq:sumk}) si creano dei vincoli che fanno sì 
	che tutti gli oggetti debbano essere più piccoli di ogni zaino,
	come spiegato nella sezione \ref{sec:extension2:Note}.
\end{itemize}
\subsubsection{Funzione obiettivo}
Lo scopo è quello di minimizzare il volume degli zaini restando nei limiti 
$S_k^\delta$. Siccome il modello deve restare lineare non è possibile calcolare
il volume degli zaini, per questo motivo si utilizza come euristica per il 
calcolo del volume la somma delle lunghezze degli spigoli $\sigma_k^\delta$:
\begin{equation}
\text{ min } \sum_{k \in K} \sum_{\delta \in \Delta} \sigma_k^\delta
\end{equation}
Siccome è desiderabile, a parità di somma delle dimensione dei vari zaini,
diminuire il numero di zaini inclusi nella soluzione è stata preferita la 
seguente funzione obiettivo:
\begin{equation}
\text{ min } \sum_{k \in K} \sum_{\delta \in \Delta} \sigma_k^\delta 
+ \sum_{k \in K} z_k
\end{equation}
\subsubsection{Vincoli}

\begin{flalign}
\sum_{\delta\in\Delta}(b_{kij}^\delta + b_{kji}^\delta) \geq t_{ki} + t_{kj} - 1
&& i < j,\ k \in K,\ i \in J,\ j \in J \\
%
\label{eq:rho:kir:leq:sigmakdelta}
\chi_{ki}^\delta + \sum_{r \in R} s_{ir}^\delta \rho_{ir} \leq \sigma_k^\delta + M \ (1-t_{ki}) && k \in K,\ i \in J,\ \delta \in \Delta \\
%
\label{eq:rho:kir:leq:precedenceA}
\chi_{ki}^\delta + \sum_{r \in R} s_{ir}^\delta \rho_{ir} \leq \chi_{kj}^\delta + M(1 - b_{kij}^\delta) && i < j,\ k \in K, i \in J, j \in J, \delta \in \Delta \\
%
\label{eq:rho:kir:leq:precedenceB}
\chi_{kj}^\delta + \sum_{r \in R} s_{jr}^\delta \rho_{jr} \leq \chi_{ki}^\delta + M(1 - b_{kji}^\delta) && i < j,\ k \in K, i \in J, j \in J, \delta \in \Delta \\
%
\chi_{ki}^\delta \leq M t_{ki} && k \in K, i \in J, \delta \in \Delta \\
%
b_{kij}^\delta \leq t_{ki} && k \in K, i \in J, j \in J, \delta \in \Delta \\
%
b_{kji}^\delta \leq t_{kj} && k \in K, i \in J, j \in J, \delta \in \Delta \\
%
\label{eq:rho:kir:sumofallrotation:tki}
\sum_{r \in R} \rho_{ir} = t_{ki} && k \in K, i \in J \\
%
\label{2:constraint:multi:tkzk}
t_{kj} \leq z_k && k \in K, j \in J \\
%
\label{2:constraint:multi:onlyInOneKnapsack}
\sum_{k \in K} t_{kj} = 1 && j \in J \\
\nonumber
%
\label{sigma_kdelta:leq:Skdelta}
\sigma_k^\delta \leq S_k^\delta z_k && \forall k \in K, \forall \delta \in \Delta \\
%
\chi_{ki}^\delta \geq 0 && k \in K, i \in J, \delta \in \Delta \\
%
t_{ki} \in \{0,1\} && k \in K, i \in J \\
%
b_{kij}^\delta \in \{0,1\} && k \in K, i \in J, j \in J, \delta \in \delta \\
%
\rho_{ir} \in \{0,1\} && k \in K, i \in J, r \in R \\
%
\label{2:zk:in:0:1}
z_k \in \{0,1\} && k \in K \\
%
\label{sigma_kdelta:geq:0}
\sigma_k^\delta \geq 0 && k\in K, \delta \in \Delta \\
\nonumber
\end{flalign}


\subsubsection{Note}
\label{sec:extension2:Note}
Vengono qui elencate brevemente le differenze con il modello descritto nella 
Sezione \ref{sec:extension1} e spiegate le motivazioni delle modifiche.
\begin{itemize}
\item Avendo esteso $\rho$ su tre dimensioni tutti i vincoli 
((\ref{eq:rho:kir:leq:sigmakdelta}), (\ref{eq:rho:kir:leq:precedenceA}), 
(\ref{eq:rho:kir:leq:precedenceB}), (\ref{eq:rho:kir:sumofallrotation:tki})) che
coinvolgono questa variabile devono essere modificati tenendo in considerazione
anche l'indice $k$.

\item Siccome il volume degli zaini in questo modello è dato dalla 
moltiplicazione di variabili, il vincolo
(\ref{eq:constraint:volume}) non può più essere considerato. 
Ma poiché è ridondante rispetto ai vincoli 
\ref{eq:constraint:coodinateLessThanSkdelta}
può essere tralasciato senza intaccare la validità del modello.
Infatti nel paper originale il vincolo (\ref{eq:constraint:volume}) viene 
considerato esclusivamente per questioni numeriche.

Al posto del vincolo (\ref{eq:constraint:coodinateLessThanSkdelta}) viene ora 
considerato il vincolo 
(\ref{eq:rho:kir:leq:sigmakdelta}) che coinvolge la variabile $\sigma_k^\delta$
invece del parametro $S_k^\delta$.

\item Il vincolo numero (\ref{constraint:multi:onlyInOneKnapsack}) deve essere 
modificato in modo tale che \emph{tutti} gli oggetti siano contenuti in uno 
zaino nella soluzione, dando forma al vincolo 
(\ref{2:constraint:multi:onlyInOneKnapsack}).

\item Il vincolo (\ref{eq:constraint:rho:ir:leq:sumk}) è stato sostituito dal
nuovo vincolo (\ref{eq:rho:kir:sumofallrotation:tki}). Qualora si tenesse il
vincolo (\ref{eq:constraint:rho:ir:leq:sumk}) si otterrebbe che qualsiasi zaino
 dovrebbe essere più grande di tutti i singoli oggetti, per via  del vincolo
 (\ref{2:constraint:multi:onlyInOneKnapsack}).

\item Inoltre, come già accennato, vanno aggiunti i bound per le variabili 
$\sigma_k^\delta$:

\begin{equation}
\sigma_k^\delta \in [ 0, S_k^\delta ]
\end{equation}

che vengono codificati nei due vincoli
(\ref{sigma_kdelta:leq:Skdelta}) e (\ref{sigma_kdelta:geq:0}).
\end{itemize}
\subsection{Dimensione del problema}

La dimensione viene qui definita come differenza rispetto a quella calcolata in
\ref{sec:orig:dimensioneDelProblema}.

Siccome si aggiungono le variabili $\sigma_k^\delta$ e le variabili $\rho_{kir}$
al posto delle variabili $\rho_{ir}$ il numero di variabili aumenta di:
\begin{itemize}
	\item $|K|\cdot|\Delta|$ unità per $\sigma_k^\delta$
	
	\item $6(|K|-1)\cdot |J|$ per $\rho_{kir}$
\end{itemize} 

Portando ad un numero approssimato usando l'assunzione $|K| < |J|$:
$$
3|J|^3 + 10|J|^2 + 4|J| 
$$


I vincoli decrescono di $|K|$ perché rimuoviamo il vincolo
(\ref{eq:constraint:volume}). E aumentano di $2 \cdot |K|\cdot|\Delta|$ per via
della definizione dei domini delle variabili $\sigma_k^\delta$ 
(i vincoli (\ref{sigma_kdelta:leq:Skdelta}) e (\ref{sigma_kdelta:geq:0})) e 
di $(|K|-1) |J|$ per via dei vincoli (\ref{eq:rho:kir:sumofallrotation:tki}).

In totale
$$
\frac{19}{2} |J|^3 + \frac{21}{2} |J|^2 + 10|J|
$$


Visto che le differenze avvengono solo al livello di grado $1$ e $2$ la
dimensione asintotica delle variabili e dei vincoli è la stessa, tuttavia il
secondo problema richiede molto più tempo per essere risolto soprattutto al
crescere di $K$ visto che deve determinare il valore di un numero di variabili
sensibilmente maggiore.


\subsection{Problema Accessorio}

Il problema accessorio è esattamente lo stesso di quello di capitolo 
\ref{sec:orig:problemaAcessorio} ma bisogna fissare anche le variabili
$\sigma_k^\delta$ e $\rho_{kir}$.



