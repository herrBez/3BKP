\section{Refusi nel paper originale}
\label{sec:refusi}
Nello sviluppo di questo lavoro sono stati riscontrati alcuni refusi nel paper 
di Baldi et al. \cite{Baldi20129802}.
\begin{itemize}
	\item Nel vincolo (\ref{eq:orig:error:10}) di \cite{Baldi20129802} c'è un errore di stampa che 
	causa in alcune istanze del problema
	la sovrapposizione di oggetti all'interno dello zaino:
	$$
	\chi_j^\delta + \sum_{r\in R} s_{jr}\rho_{ir} \leq \chi_i^\delta + M(1-b_{ji})
	$$
	dovrebbe essere 
	$$
	\chi_j^\delta + \sum_{r\in R} s_{jr}\rho_{\mathbf{\color{red}j}r} \leq \chi_i^\delta + M(1-b_{ji})
	$$
	Infatti sostituendo la prima equazione con la seconda, tutte le 
	sovrapposizioni tra gli oggetti scompaiono.
	
	\item Il vincolo (\ref{eq:orig:chi+s_ir:leq:SdeltaK}) unito al vincolo 
	(\ref{eq:orig:constraint:rho:ir:leq:sumk})
	di \cite{Baldi20129802}, 
	fanno sì che, qualora si inserisca nell'insieme $J$ un oggetto che non abbia
	abbastanza spazio all'interno dello zaino, il modello sia inammissibile.
	Per ovviare a questo problema si può modificare il vincolo 
	(\ref{eq:orig:constraint:rho:ir:leq:sumk}) 
	di \cite{Baldi20129802} nel modo seguente:
	$$
	\sum_{r \in R} \rho_{ir} = \color{red}\mathbf{t_i}
	$$
	
	
	Infatti l'oggetto deve avere una rotazione soltanto se viene effettivamente
	inserito nello zaino.
	Una soluzione alternativa a questo problema potrebbe consistere nel 
	controllare che tutti gli oggetti siano più piccoli dello zaino, 
	prima di istanziare il modello.
	La prima soluzione è stata preferita alla seconda perché si prestava meglio
	alle estensioni mostrate in questo documento.
	\item Il vincolo (\ref{eq:orig:constraint:bij:leq:ti}) e 
	(\ref{eq:orig:constraint:bji:leq:tj}) di \cite{Baldi20129802} sono ridondanti, 
	infatti ogni vincolo in (\ref{eq:orig:constraint:bij:leq:ti}) ha un 
	corrispettivo in (\ref{eq:orig:constraint:bji:leq:tj}). L'unica differenza
	è che i vincoli vengono elencati in due maniere diverse. 
\end{itemize}

