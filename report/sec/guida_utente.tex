\section{Come eseguire il programma}
\subsection{Compilazione}
La cartella \verb|Multi3BKP| contiene l'implementazione del primo modello, 
mentre \verb|Hull3BKP| contiene l'implementazione del problema alternativo.
Per compilare è sufficiente utilizzare il \verb|Makefile| contenuto nella 
rispettiva cartella.


\subsection{File di Input}
Per comodità entrambi i programmi utilizzano lo stesso file di input, anche se 
non tutti i parametri vengono utilizzati in tutti e due i programmi.
\begin{itemize}
	\item $K$, il numero di zaini
	\item Seguono $K$ righe contenente nella riga $k$ i dati per lo zaino $k$ 
	intervallati da uno spazio:
	\begin{itemize}
		\item $S_k^0$
		\item $S_k^1$
		\item $S_k^2$
		\item $f_k$
	\end{itemize} 
	Se il problema è esteso (con i \emph{balancing constraints}) si 
	aggiungono anche:
	\begin{itemize}
		\item $L_k^0$
		\item $L_k^1$
		\item $L_k^2$
		\item $U_k^0$
		\item $U_k^1$
		\item $U_k^2$
	\end{itemize}
	\item $N$, il numero di oggetti ($|J|$)
	\begin{itemize}
		\item $s_i^1$
		\item $s_i^2$
		\item $s_i^3$
		\item $m_i$
		\item $p_i$
	\end{itemize}
\end{itemize}

\subsection{File di output}
In questa sezione siano $K' \subseteq K$ l'insieme di zaini selezionati, 
$J^k$ gli oggetti che sono stati inseriti nel $k$--simo zaino e sia $r'$ la 
rotazione selezionata per l'oggetto $i$.


Il programma genera un file di output così costituito:
\begin{itemize}
	\item Una serie di righe di commento che includono anche il valore ottimo 
	trovato
	\item $|K'|$ blocchi così formati:
		\begin{itemize}
			\item \verb|# Knapsack k-mo|
			\item $S_k^\delta$ o $\sigma_k^\delta$ a seconda se il programma è 
			stato generato con \verb|main| o \verb|main_hull|
			\item Seguono $|J^k|$ righe contenenti le seguenti informazioni 
			sugli oggetti inseriti nello zaino $k$:
			\begin{itemize}
				\item $i$, l'indice dell'oggetto
				\item $\chi_{ki}^0$
				\item $\chi_{ki}^1$
				\item $\chi_{ki}^2$
				\item $s_{ir'}^0$
				\item $s_{ir'}^1$
				\item $s_{ir'}^2$
			\end{itemize}
		\end{itemize}
\end{itemize}






\subsection{Visualizzazione}
Per visualizzare il file di output ed agevolare le fasi di verifica e 
validazione del modello è stato implementato un programma che rappresenta gli 
zaini e gli oggetti selezionati nello spazio, utilizzando la libreria grafica 
\verb|webgl|\footnote{\url{https://www.khronos.org/webgl/}}.

\subsection{Documentazione}
La documentazione del codice può essere generata utilizzando il programma 
\verb|doxygen|\footnote{\url{https://github.com/doxygen/doxygen}} 
all'interno delle rispettive cartelle.

