\section{Introduzione}
\setcounter{equation}{4}

Lo scopo di questo documento è proporre alcune possibili estensioni per il 
modello esatto definito in Baldi et al. \cite{Baldi20129802}, 
che d'ora in poi verrà riferito come \emph{paper originale}.

\begin{problem}[3KP]
Il problema consiste nel posizionare ortogonalmente un sottoinsieme di 
oggetti tridimensionali all'interno di uno zaino in modo tale da 
massimizzare il profitto totale degli oggetti caricati. 
Gli oggetti posizionati non devono avere sovrapposizioni e possono 
essere ruotati.
\end{problem}

\begin{problem}[3BKP]
Estensione del 3KP in cui il centro di massa dello zaino deve risiedere
all'interno di un'area specificata dello zaino (vincoli di bilanciamento).
\end{problem}

Nella Sezione \ref{sec:modello:originale} viene riportato il modello descritto 
nel paper originale~\cite{Baldi20129802} per rendere i riferimenti al modello
di partenza più immediati e quindi il documento più leggibile.

Nella Sezione \ref{sec:extension1} verrà analizzata un'estensione del 3KP con 
più zaini, cui viene associato un costo fisso differente per ciascuno zaino.

Nella Sezione \ref{sec:extension2} viene considerato un problema analogo, 
ma differente il cui scopo è quello di inserire tutti gli oggetti all'interno 
degli zaini e cercare di minimizzare le dimensioni e il numero degli stessi. 

All'interno della Sezione \ref{sec:Balancing:Constraint} viene descritto come 
si potrebbero estendere i vincoli di bilanciamento per più zaini. 


Seguono una sezione in cui si descrive come compilare il 
programma e i formati dei file di input e di output e una sezione in cui
vengono evidenziati alcuni refusi 
del paper originale riscontrati durante lo sviluppo di questo progetto.

Per concludere segue una una discussione sull'efficienza dei vari
modelli a trattare istanze generate casualmente.
I risultati del benchmark sono mostrati nell'Appendice~\ref{sec:results}.

