\section{Balancing Constraints}
\label{sec:Balancing:Constraint}
È stata anche considerata la possibilità di estendere i balancing constraint 
del paper originale:
\newcounter{tmp}
\setcounter{tmp}{\value{equation}}
\setcounter{equation}{13}
\begin{equation}
\label{eq:orig:balancing1}
\sum_{i \in J} m_i \chi_{i}^\delta + \sum_{i \in J}\sum_{r \in R} m_i \gamma_{ir}^\delta \rho_{ir} \geq L^\delta \sum_{i \in J} m_i t_i \quad\quad \delta \in \Delta
\end{equation}
\begin{equation}
\label{eq:orig:balancing2}
\sum_{i \in J} m_i \chi_{i}^\delta + \sum_{i \in J}\sum_{r \in R} m_i \gamma_{ir}^\delta \rho_{ir} \leq U^\delta \sum_{i \in J} m_i t_i \quad\quad \delta \in \Delta
\end{equation}
\setcounter{equation}{\value{tmp}}
dove 
\begin{itemize}
	\item $\gamma_{ir}^\delta$ è la coordinata del centro di massa dell'oggetto 
	$i$ nella rotazione $r$ nella dimensione $\delta$.
	\item $L^\delta$ è il lower bound 
	\item $U^\delta$ è l'upper bound
\end{itemize}

per il modello con più zaini:
\begin{equation}
\sum_{i \in J} m_i \chi_{ki}^\delta + \sum_{i \in J}\sum_{r \in R} m_i \gamma_{ir}^\delta \rho_{ir} \geq L_k^\delta \sum_{i \in J} m_i t_{ki} \quad\quad k \in K, \delta \in \Delta
\end{equation}
\begin{equation}
\sum_{i \in J} m_i \chi_{ki}^\delta + \sum_{i \in J}\sum_{r \in R} m_i \gamma_{ir}^\delta \rho_{ir} \leq U_k^\delta \sum_{i \in J} m_i t_{ki} \quad\quad k \in K, \delta \in \Delta
\end{equation}

In questo caso vengono aggiunti $2 \cdot (|K| \cdot 3)$ vincoli a quelli del 
modello di partenza.

Nel problema alternativo, cosí come è definito, i balancing constraints non 
sono stati considerati, perchè i parametri $L$ e $U$ dovrebbero essere a loro
volta delle variabili (dipendenti dalle variabili $\sigma_{k}^{\delta}$) e quindi
il modello diverrebbe \emph{non} lineare e, visto che le dimensioni dello zaino
non sono prevedibili a priori il significato (concreto) di questi vincoli verrebbe meno:
perchè sono stati considerati per permettere un carico bilanciato dei pesi all'interno
di parametri $U$ e $L$ fissati.