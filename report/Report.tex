\documentclass{scrartcl}

\usepackage[utf8]{inputenc}
\usepackage[italian]{babel}
\begin{document}
\section{Note}
Il modello è stato implementato (per ora) tralasciando i vincoli del centro di massa.
Nella prossima sezione segue una descrizione di come possano essere aggiunti.
\paragraph{Formato di file}
Il formato di input è il seguente:
\begin{itemize}
	\item la prima riga contiene le dimensioni della scatola $(S^1, S^2, S^3) = (W, D, H)$
	\item la seconda contiene $N$ il numero di ogetti
	\item Seguono $N$ righe con il seguente formato:
	\begin{itemize}
		\item $s^0$ nella rotazione numero $0$
		\item $s^1$ nella rotazione numero $0$
		\item $s^2$ nella rotazione numero $0$
		\item Massa
		\item Profitto
	\end{itemize}
\end{itemize}

\paragraph{Note sui constraint}
\begin{itemize}
	\item I vincoli 9 e 10 richiedono l'utilizzo di un $M$. Per questo
	valore è stato scelto $10^6$, ma probabilmente con altre istanze il numero deve essere cambiato.
	
	\item Di seguito vengono riportate alcune modifiche ai constraint in modo da portare
	tutte le variabili nella parte sinistra della dis/equazione.
	\begin{itemize}
	\item 
	Il constraint numero (7) è stato riscritto come:
	$$
	\begin{array}{l}
	\sum_{\delta \in \Delta}(b_{ij}^\delta + b_{ji}^\delta) \geq t_i + t_j - 1 \iff \\
	\sum_{\delta \in \Delta}(b_{ij}^\delta + b_{ji}^\delta) - t_i - t_j \geq - 1 \iff \\
	+ b_{ij}^1 + b_{ji}^1 + b_{ij}^2 + b_{ji}^2 + b_{ij}^3 + b_{ji}^3 - t_i - t_j \geq  -1
	\end{array}
	$$
	\item 
	Il vincolo numero (9) è stato riscritto come:
	$$
	\begin{array}{l}
	\chi_i^\delta + \sum_{r \in R} s_{ir}^\delta \rho_{ir} \leq \chi_j^\delta + M(1 - b_{ij}^\delta)\\
	\chi_i^\delta + (\sum_{r \in R} s_{ir}^\delta \rho_{ir}) -  \chi_j^\delta + M b_{ij}^\delta \leq M
	\end{array}
	$$
	\item 
	Il vincolo numero (10) è stato riscritto come:
	$$
	\begin{array}{l}
	\chi_j^\delta + \sum_{r \in R} s_{jr}^\delta \rho_{ir} \leq \chi_j^\delta + M(1 - b_{ji}^\delta)\\
	\chi_j^\delta + (\sum_{r \in R} s_{jr}^\delta \rho_{ir}) -  \chi_i^\delta + M b_{ji}^\delta \leq M
	\end{array}
	$$
\end{itemize}
\end{itemize}
\section{Per aggiungere i balancing constraint (14/15)}
Per aggiungere i vincoli del centro di massa bisogna:
\begin{itemize}
	\item implementare il metodo \verb|calculateGamma| nella classe \verb|instance3BKP|.
	Questo metodo dovrebbe calcolare i valori di $\gamma_{ir}^\delta$.
	\item Aggiungere ai file delle istanze due righe contenenti $L^0, L^1, L^2$ e $U^0, U^1, U^2$
	\item Usare l'opzione \verb|--extended| o per brevità \verb|-e|.
\end{itemize}


\section{Istanze}
Di seguito commento le istanze
\begin{table}[h!]
\begin{tabular}{|l|l|l|}
	\hline
	\textbf{Istanza} & \textbf{Funziona} & \textbf{Commento} \\
	\hline
	\verb|es438911.dat| & SÍ & \\
	\hline
	\verb|es595423.dat| & NO & Almeno tre intersezioni \\
	\hline
	\verb|es377399.dat| & NO & Almeno tre intersezioni \\
	\hline
\end{tabular}
\end{table}

\end{document}
