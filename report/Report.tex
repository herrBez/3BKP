\documentclass{scrartcl}

\usepackage[utf8]{inputenc}
\usepackage[italian]{babel}
\usepackage{amsmath}
\begin{document}

\section{Note}

Nella prossima sezione segue una descrizione di come possano essere aggiunti.
\paragraph{Formato di file}
Il formato di input è il seguente:
\begin{itemize}
	\item la prima riga contiene le dimensioni della scatola $(S^1, S^2, S^3) = (W, D, H)$
	\item la seconda contiene $N$ il numero di ogetti
	\item Seguono $N$ righe con il seguente formato:
	\begin{itemize}
		\item $s^0$ nella rotazione numero $0$
		\item $s^1$ nella rotazione numero $0$
		\item $s^2$ nella rotazione numero $0$
		\item Massa
		\item Profitto
	\end{itemize}
\end{itemize}

\paragraph{Note sui constraint}
\begin{itemize}
	\item I vincoli 9 e 10 richiedono l'utilizzo di un $M$. Per questo
	valore è stato scelto $10^6$, ma probabilmente con altre istanze il numero deve essere cambiato.
	
	\item Di seguito vengono riportate alcune modifiche ai constraint in modo da portare
	tutte le variabili nella parte sinistra della dis/equazione.
	\begin{itemize}
	\item 
	Il constraint numero (7) è stato riscritto come:
	$$
	\begin{array}{l}
	\sum_{\delta \in \Delta}(b_{ij}^\delta + b_{ji}^\delta) \geq t_i + t_j - 1 \iff \\
	\sum_{\delta \in \Delta}(b_{ij}^\delta + b_{ji}^\delta) - t_i - t_j \geq - 1 \iff \\
	+ b_{ij}^1 + b_{ji}^1 + b_{ij}^2 + b_{ji}^2 + b_{ij}^3 + b_{ji}^3 - t_i - t_j \geq  -1
	\end{array}
	$$
	\item 
	Il vincolo numero (9) è stato riscritto come:
	$$
	\begin{array}{l}
	\chi_i^\delta + \sum_{r \in R} s_{ir}^\delta \rho_{ir} \leq \chi_j^\delta + M(1 - b_{ij}^\delta)\\
	\chi_i^\delta + (\sum_{r \in R} s_{ir}^\delta \rho_{ir}) -  \chi_j^\delta + M b_{ij}^\delta \leq M
	\end{array}
	$$
	\item 
	Il vincolo numero (10) è stato riscritto come:
	$$
	\begin{array}{l}
	\chi_j^\delta + \sum_{r \in R} s_{jr}^\delta \rho_{ir} \leq \chi_j^\delta + M(1 - b_{ji}^\delta)\\
	\chi_j^\delta + (\sum_{r \in R} s_{jr}^\delta \rho_{ir}) -  \chi_i^\delta + M b_{ji}^\delta \leq M
	\end{array}
	$$
\end{itemize}
\end{itemize}
\section{Per aggiungere i balancing constraint (14/15)}
Per aggiungere i vincoli del centro di massa bisogna:
\begin{itemize}
	\item Aggiungere ai file delle istanze due righe contenenti $L^0, L^1, L^2$ e $U^0, U^1, U^2$
	\item Usare l'opzione \verb|--extended| o per brevità \verb|-e|.
\end{itemize}


\section{Sviste del paper}
\begin{itemize}
	\item Nel vincolo numero 10 $\rho$ non usa i giusti indici
\end{itemize}

\subsection{Oggetti più grandi dello zaino}
Si prenda in considerazione un problema dello zaino in 3 dimensioni in cui:
\begin{itemize}
	\item $(S^0,S^1,S^2) = (1,1,1)$
	\item $J = \{1\}$
	\item $(s_{1, 0}^0,s_{1,0}^1, s_{1,0}^2) = (2,1,1)$
\end{itemize}
Una volta inseriti i vincoli abbiamo nel vincolo 8:
$$
\begin{array}{ll}
\chi_0^0 + 2 \rho_{00} + 2 \rho_{01} + \rho_{02} + \rho_{03} + \rho_{04} + \rho_{05} \leq 1 & \\
\chi_1^0 +  \rho_{00} +  \rho_{01} + 2 \rho_{02} + \rho_{03} + 2 \rho_{04} + \rho_{05} \leq 1 & \\
\chi_0^0 +  \rho_{00} +  \rho_{01} + \rho_{02} + 2 \rho_{03} + \rho_{04} + 2 \rho_{05} \leq 1 & \\
\end{array}
$$
Ora siccome deve valere il vincolo numero $(16)$ abbiamo:
$$
\sum_{r\in R} \rho_{ir} = 1 \ \ \ \forall i \in J
$$
ma visto che le variabili sono binarie:
$$
\sum_{r\in R} \rho_{ir} = 1 \iff \exists r \in R : \rho_{ir} = 1
$$
anche per gli oggetti che non vengono inseriti nello zaino.
Quindi visto che le variabili $\chi$ sono positivi nell'esempio si ha che, se:
$$
\begin{array}{l}
\rho_{00} = 1 \implies \chi_1^0 + 2 \rho_{00} > 1 \\
\rho_{01} = 1 \implies \chi_1^0 + 2 \rho_{01} > 1 \\
\rho_{02} = 1 \implies \chi_1^1 + 2 \rho_{02} > 1 \\
\rho_{03} = 1 \implies \chi_1^2 + 2 \rho_{03} > 1 \\
\rho_{04} = 1 \implies \chi_1^1 + 2 \rho_{04} > 1 \\
\rho_{05} = 1 \implies \chi_1^2 + 2 \rho_{05} > 1 \\
\end{array}
$$

\subsection{Soluzioni}
\begin{itemize}
	\item aggiungere una "settima" rotazione in cui tutti i valori $s_*$ sono nulli.
	Per esempio per evitare troppi \verb|if-else| si potrebbe aggiungere una quarta dimensione fittizia ai vari $s$ e aggiungere come settima rotazione la tripletta ${4,4,4}$.
	
\end{itemize}
\end{document}
